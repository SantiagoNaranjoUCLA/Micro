\section{Housing Prices, in Practice}

\begin{tcolorbox}
    It is commonly believed by the general public that an increase in housing supply can increase prices.
    Figure 1 shows a Tweet about one case in Boston. Figure 2 presents a survey that shows people
    generally think increasing supply lowers prices, except when it comes to housing.\\

    Does the argument in Figure 1 have merit? What does this argument overlook? How does this
    mesh with the traditional supply-demand analysis? [1 page maximum]
\end{tcolorbox}

It is a well-known principle of conventional supply and demand analysis that, under normal circumstances, higher supply leads to lower prices. The housing price paradox, however, casts doubt on this theory by arguing that a greater concentration of homes in a given region can raise demand for those homes.\\

We may examine the home price paradox by taking into account the concept of perception bias in market dynamics as the housing market and other markets share fundamental economic principles. The peculiarities of the housing market, in contrast to other markets, add to this paradox. Notably, it exhibits slower price adjustments, creating a time lag from a change in value to its reflection in prices. This time delay makes it challenging for participants to have timely and accurate knowledge of an asset's fair value. Additionally, the housing market features diverse assets, and identical houses may have significantly different prices due to various external factors.\\

As seen in Figure 2, people are generally good at predicting how changes in supply will affect pricing; however, as market complexity rises, this capacity appears to decline. A bias in perception regarding the variables influencing home prices is produced by the combination of asset heterogeneity and price adjustment time lag. Market players often use the assumption that correlation equates to causality in order to simplify price changes. The notion that, for instance, a new, more costly property will increase area value is an example of linear thinking that ignores basic economic principles.\\

In reality, more costly homes might, in fact, act more as triggers to bring prices down to market level than as price leaders. Due to suppliers' perception that they can sell items for more than their cost of production, the adjustment impact naturally tends to the upward direction. This is not because expensive houses lead the market but because they accelerate the market's adjustment to a new equilibrium.\\
