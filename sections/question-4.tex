\section{Housing Prices, in Practice}

\begin{tcolorbox}
    It is commonly believed by the general public that an increase in housing supply can increase prices.
    Figure 1 shows a Tweet about one case in Boston. Figure 2 presents a survey that shows people
    generally think increasing supply lowers prices, except when it comes to housing.\\

    Does the argument in Figure 1 have merit? What does this argument overlook? How does this
    mesh with the traditional supply-demand analysis? [1 page maximum]
\end{tcolorbox}

In traditional supply and demand analysis, it's commonly understood that an increase in supply, all else being equal, tends to bring down prices. However, the housing price paradox challenges this intuition by suggesting that more houses in a particular area could lead to increased demand for those houses.\\

Recognizing that the housing market shares fundamental economic principles with other markets, we can delve into the housing price paradox by considering the idea of perception bias in market dynamics. Unlike some markets, the housing market has unique characteristics that contribute to this paradox. Notably, it exhibits slower price adjustments, creating a time lag from a change in value to its reflection in prices. This time delay makes it challenging for participants to have timely and accurate knowledge of an asset's fair value. Additionally, the housing market features diverse assets, and identical houses may have significantly different prices due to various external factors.\\

As illustrated in Figure 2, individuals generally predict how changes in supply impact prices well, but this ability seems to diminish as market complexity increases. The combination of price adjustment time lag and asset heterogeneity creates a perception bias regarding the factors influencing home prices. Market participants tend to simplify price changes by implying that correlation implies causation. For example, the belief that a new, more expensive property would drive up area value represents linear thinking that overlooks fundamental economic insights.\\

In reality, new, more expensive houses may not serve as price leaders but rather as catalysts that expedite price adjustments to the market level. Since suppliers produce based on the belief they can sell goods at a price exceeding production costs, the adjustment effect inherently trends upward. This is not because expensive houses lead the market but because they accelerate the market's adjustment to a new equilibrium.\\
