\section{Housing Prices, in Practice}

\begin{tcolorbox}
    It is commonly believed by the general public that an increase in housing supply can increase prices.
    Figure 1 shows a Tweet about one case in Boston. Figure 2 presents a survey that shows people
    generally think increasing supply lowers prices, except when it comes to housing.\\

    Does the argument in Figure 1 have merit? What does this argument overlook? How does this
    mesh with the traditional supply-demand analysis? [1 page maximum]
\end{tcolorbox}

In traditional Supply - Demand analysis, an increase in supply, ceteris paribus, tends to lower prices. The housing price paradox, as perceived by the public, fundamentally questions the intuition of this perspective by arguing that an increase in the number of houses in a particular area would subsequently lead to increased demand for these houses. \\

By acknowledging that the housing market is not a peripheral market with unique characteristics which could justify deviations from conventional economic analysis, we will approach this narrative and attempt to reason through it by leveraging the idea of perception bias about how markets work. \\

While the housing market shares fundamental economic principles with other markets, it possesses unique characteristics that contribute to the housing price paradox. Unlike commodities or the stock market, the housing market experiences slower adjustments in prices, introducing a time lag from the moment of a change in value to its reflection in prices. Consequently, it becomes challenging for market participants to have timely and accurate knowledge of the fair value of an asset at any given time. Another distinctive feature of the housing market is the heterogeneity of assets. Unlike an ounce of gold, which remains identical regardless of the exchange (NYSE or LSE), two identical houses may have significantly different prices due to a series of externalities.\\

As Figure 2 suggests, agents are generally good at predicting how changes in supply impact prices, but this ability appears to diminish as the complexity of the market increases. The price adjustment time lag combined with the heterogeneity of assets creates a perception bias regarding the factors that drive home prices. Agents tend to rationalize and simplify price changes by unconsciously implying that correlation implies causation. In other words, the belief that a new, more expensive property would drive up area value represents linear thinking that overlooks a fundamental economic insight. Suppliers produce as long as they believe they can sell the goods for a price that exceeds their marginal cost of production. This implies that, in reality, new, more expensive houses don’t serve as price leaders, but rather as catalysts that accelerate the adjustment of prices to the market level, and since it wouldn’t make economic sense for a company to develop construction  at a time when prices are falling, the adjustment effect will inherently trend upward.\\
