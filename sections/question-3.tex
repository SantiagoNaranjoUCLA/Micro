\section{Housing Prices, in Theory}

A unit mass of people indexed by \( v \sim U[0,1] \) must choose to live in either Los Angeles or Kansas City. Each city has housing stock \( 3/4 \), so there is enough housing between the two cities but not in any one city. House prices are determined by a competitive market of landlords; they have no costs if they rent out their house.\\

Agents have utilities \( u_{LA} = b + v - p_{LA} \) and \( u_{KC} = b - p_{KC} \) from living in the two cities, where \( p_{LA} \) and \( p_{KC} \) are the prices of renting in the two cities, \( v \sim U[0,1] \) indicates how much the agent likes California weather, and \( b \) is the benefit of having a house (we assume this is positive, so no-one chooses to be homeless).



%%%%%%%%%%%%%%%%%%%%%%%%%%%%%%%%%%%%%%%%%%%%%%%%%%%%%%%%%%%%%%%%%
\begin{tcolorbox}
    (a) Suppose \( p_{LA} = p_{KC} = 0 \). Is this an equilibrium? Is there excess demand/supply in either market?
\end{tcolorbox}

When \( p_{LA} = p_{KC} = 0 \):

\begin{eqnarray*}
    \mathbb{E}[u_{LA}] &=& 0.5 + b\\
    \mathbb{E}[u_{KC}] &=& b\\
    \mathbb{E}[u_{LA}] &>& \mathbb{E}[u_{KC}]
\end{eqnarray*}

Furthermore:

\begin{eqnarray*}
    Pr(v = 0) &=& 0\\
\end{eqnarray*}


Because \(v\) is a continuous random variable distributed uniformly between zero and one, the probability of \(v\) being exactly zero is zero.\\

Therefore, for every agent, it is preferable to live in Los Angeles, as it is highly unlikely that they have a null preference for the climate of Los Angeles.\\

\begin{myanswerbox}
    In this case, the excess demand for housing in Los Angeles is \(1/4\), being this the proportion of the population that must live in Kansas City.\\

    This is not an equilibrium because Los Angeles landlords can raise the price of their properties to a point where \(\mathbb{E}[u_{LA}] = \mathbb{E}[u_{KC}]\) and thus maximize their profits.
\end{myanswerbox}

The demand for housing in Los Angeles depends on the proportion of agents who prefer to live there rather than in KC, given the price \( p_{LA} \). This proportion is given by:


\begin{eqnarray*}
    Pr(u_{LA} &>& u_{KC})\\
    Pr(v + b - p_{LA} &>& b - p_{KC})\\
    Pr(v &>& p_{LA} - p_{KC})\\
\end{eqnarray*}

\begin{equation}
    D_{LA} = Pr(v > p_{LA} - p_{KC})\\
\end{equation}

\begin{equation}
    D_{LA} = \begin{cases}
        0, &  p_{LA} - p_{KC} \geq 1\\
        1 - (p_{LA} - p_{KC}), & 0 < p_{LA} - p_{KC} < 1\\
        1, &  p_{LA} - p_{KC} \leq 0\\
    \end{cases}
\end{equation}

And the demand for housing in Kansas City is:

\begin{equation}
    D_{KC} = \begin{cases}
        0, &  p_{LA} - p_{KC} \leq 0\\
        (p_{LA} - p_{KC}), & 0 < p_{LA} - p_{KC} < 1\\
        1, &  p_{LA} - p_{KC} \geq 1\\
    \end{cases}
\end{equation}

The supply function for housing in Los Angeles and Kansas City is:

\begin{eqnarray}
    S_{LA}(p_{LA}) &=& 3/4\\
    S_{KC}(p_{KC}) &=& 3/4
\end{eqnarray}
%%%%%%%%%%%%%%%%%%%%%%%%%%%%%%%%%%%%%%%%%%%%%%%%%%%%%%%%%%%%%%%%%
%%%%%%%%%%%%%%%%%%%%%%%%%%%%%%%%%%%%%%%%%%%%%%%%%%%%%%%%%%%%%%%%%
\begin{tcolorbox}
    (b) What are the equilibrium prices \( p_{LA}, p_{KC} \)?
\end{tcolorbox}

Assuming that landlords in Los Angeles do not know the preferences of the agents (since \( v \) is a random variable), their best strategy is to choose a price such that:

\begin{eqnarray*}
    \mathbb{E}[u_{LA}] &=& \mathbb{E}[u_{KC}]\\
    0.5 + b - p_{LA} &=& b - p_{KC}\\
\end{eqnarray*}

\begin{equation}
    p_{LA} = 0.5 + p_{KC}
    \label{eq:equilibrium}
\end{equation}

In this case, the exercise of housing choice is equivalent to a Monte Carlo simulation where the proportion of agents living in LA will be the same as those living in KC, and there will be no incentives for LA landlords to raise prices.\\

Since the marginal cost of renting is zero, and given that landlords in Los Angeles compete with those in Kansas City, the minimum possible price for those in Kansas City is \( p_{KC} = 0 \), making them indifferent between renting or not.


\begin{myanswerbox}
    The equilibrium prices will be:
    \begin{eqnarray*}
        p_{LA} &=& 0.5\\
        p_{KC} &=& 0
    \end{eqnarray*}

    Thus, the expected proportion of agents living in Los Angeles and Kansas City will be 50\% for each city. Additionally, the landlords in Kansas City will be indifferent between renting or not.\\
\end{myanswerbox}

The expected proportion of agents demanding housing in LA and KC is: 

\begin{eqnarray*}
    \mathbb{E}[D_{LA}] &=& \mathbb{E}[D_{KC}] = 0.5\\
\end{eqnarray*}
    
The expected occupancy in LA and KC is 66.67\% (with \( \frac{3}{4} \) being 100\%), which means that this is the expected proportion of landlords who will be generating profits from renting their properties in LA. From the total housing supply (1.5 adding LA and KC), 25\% is expected to generate profits (3/8 in LA of the total offered equivalent to 1.5). \\


The expected surplus of the producers is: 


\begin{eqnarray*}
    \mathbb{E}[PS] &=& \mathbb{E}[PS_{LA}] + \mathbb{E}[PS_{KC}]\\
    \mathbb{E}[PS] &=& (1/2*0.5 + 1/4 * 0) + (1/4 * 0 + 1/2 * 0)\\
    \mathbb{E}[PS] &=& 0.25\\
\end{eqnarray*}

From the demand curves, it is possible to calculate the consumer surplus in each city:

\begin{eqnarray*}
    CS_{LA} &=& \int_{0.5}^{1}  D_{LA}(0.5) dx\\
    CS_{LA} &=& \int_{0.5}^{1}  (1 - 0.5)dx\\
    CS_{LA} &=& 0.25\\
\end{eqnarray*}

Similarly, the consumer surplus in Kansas City is:

\begin{eqnarray*}
    CS_{KC} &=& \int_{0.5}^{1}  D_{KC}(0.5) dx\\
    CS_{KC} &=& \int_{0.5}^{1}  0.5dx\\
    CS_{KC} &=& 0.25\\
\end{eqnarray*}

These surpluses should be interpreted as the probability that agents have a utility higher than expected.\\

Therefore, the total welfare is:

\begin{eqnarray*}
    SW &=& CS_{LA} + CS_{KC} + \mathbb{E}[PS]\\
    SW &=& 0.25 + 0.25 + 0.25\\
    SW &=& 0.75\\
\end{eqnarray*}

%%%%%%%%%%%%%%%%%%%%%%%%%%%%%%%%%%%%%%%%%%%%%%%%%%%%%%%%%%%%%%%%%
%%%%%%%%%%%%%%%%%%%%%%%%%%%%%%%%%%%%%%%%%%%%%%%%%%%%%%%%%%%%%%%%%
\begin{tcolorbox}
    (c) What happens to house prices in LA and KC if we build a few more houses in LA or KC?
\end{tcolorbox}

\begin{myanswerbox}
    Since the addition of housing in either city does not change the cost structure for renting and the preference of agents, equation \ref{eq:equilibrium} remains valid, and therefore, the prices will be the same as in the previous clause.
\end{myanswerbox}
%%%%%%%%%%%%%%%%%%%%%%%%%%%%%%%%%%%%%%%%%%%%%%%%%%%%%%%%%%%%%%%%%
%%%%%%%%%%%%%%%%%%%%%%%%%%%%%%%%%%%%%%%%%%%%%%%%%%%%%%%%%%%%%%%%%
\begin{tcolorbox}
    (d) Suppose we can build new housing at constant marginal cost \( c = 1/10 \) in both cities. What is the long-run equilibrium price? How many people live in each city?
\end{tcolorbox}

The population mass is still indexed by \( v \), so the equilibrium condition \( \mathbb{E}[u_{LA}] = \mathbb{E}[u_{KC}] \) remains valid. However, the minimum price at which Kansas City landlords can rent is \( c \). At that price level, the demand for housing in Kansas City is \( D_{KC} = p_{LA} - c \), and the demand in LA is \( D_{LA} = 1 - (p_{LA} - c). \)\\

Given the equilibrium relationship: 

    \begin{equation*}
    p_{LA} = 0.5 + p_{KC}
\end{equation*}

\begin{myanswerbox}
    
    Therefore, the new set of prices is \( p_{LA} = 0.6 \) and \( p_{KC} = 0.1 \).\\

    The expected demand for housing in Los Angeles and Kansas City is:

    \begin{eqnarray*}
        D_{KC} &=& p_{LA} - c = 0.6 - 0.1 = 0.5\\
        D_{LA} &=& 1 - (p_{LA} - c) = 1 - (0.6 - 0.1) = 0.5\\
    \end{eqnarray*}

\end{myanswerbox}


%%%%%%%%%%%%%%%%%%%%%%%%%%%%%%%%%%%%%%%%%%%%%%%%%%%%%%%%%%%%%%%%%
%%%%%%%%%%%%%%%%%%%%%%%%%%%%%%%%%%%%%%%%%%%%%%%%%%%%%%%%%%%%%%%%%
\begin{tcolorbox}
    (e) How much does social welfare increase from the extra building in (d)?
\end{tcolorbox}

\begin{eqnarray*}
    CS_{LA} &=& \int_{p_{LA} - p_{KC}}^{1}  D_{LA}(p_{LA} - p_{KC}) dx\\
    CS_{LA} &=& \int_{0.5}^{1}  (1 - 0.5)dx\\
    CS_{LA} &=& 0.25\\
\end{eqnarray*}

Similarly, the consumer surplus in Kansas City is:

\begin{eqnarray*}
    CS_{KC} &=& \int_{p_{LA} - p_{KC}}^{1}  D_{KC}(p_{LA} - p_{KC}) dx\\
    CS_{KC} &=& \int_{0.5}^{1}  0.5dx\\
    CS_{KC} &=& 0.25\\
\end{eqnarray*}

The producer surplus is given by those landlords who are able to rent their properties. Due to the equilibrium condition and given that we are analyzing the long run, the expected supply of housing in each city is equal to the expected demand. Additionally, those landlords in Kansas City will rent at a price of \( c \), making no profits.\\

\begin{eqnarray*}
    \mathbb{E}[PS] &=& \mathbb{E}[PS_{LA}] + \mathbb{E}[PS_{KC}]\\
    \mathbb{E}[PS] &=& (1/2*0.5) + (1/2 * 0)\\
    \mathbb{E}[PS] &=& 0.25\\
\end{eqnarray*}

Therefore, the total welfare is:

\begin{eqnarray*}
    SW &=& CS_{LA} + CS_{KC} + \mathbb{E}[PS]\\
    SW &=& 0.25 + 0.25 + 0.25\\
    SW &=& 0.75\\
\end{eqnarray*}

\begin{myanswerbox}
    As the economy adjusts in the long term, the excess supply causes those suppliers who do not rent out their properties to incur losses. Due to this, they decide to exit the market until the supply adjusts to the expected demand. As the supply adjusts, the producers' surplus increases until it reaches the level calculated in section (b), where the marginal cost of renting was zero.\\

    The Social Welfare is calculated as \( SW = 0.75 \). There is no increase in social welfare because the probability of agents to exceed their expected utility is the same. Also, those landlords who remain in the market are making the same profits as in the case where the marginal cost of renting was zero.\\
\end{myanswerbox}

%%%%%%%%%%%%%%%%%%%%%%%%%%%%%%%%%%%%%%%%%%%%%%%%%%%%%%%%%%%%%%%%%